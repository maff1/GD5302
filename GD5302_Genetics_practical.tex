\documentclass[a4paper,11pt]{article}

\usepackage{fullpage}
\usepackage{palatino}
\usepackage{url}
\usepackage{listings}

\begin{document}

\title{Practical 3: Genetics Practical}
\author{GD5302: Digital Health Practice}
\date{Due date: Tuesday 16th April 2024 (Week 13) at 12:00 (midday)\\
25\% of the coursework grade.\\
Please note that MMS is definitive for weighting and deadlines, which, occasionally have to be changed.} 

\maketitle

\subsection*{Aims}

The main aim of this practical is to XXXX. 
You will (1 OR 2 SENTENCE OUTLINE OF WHAT THEY WILL DO). 

\subsection*{Task}
DESCRIPTION OF TASK

(IF HELPFUL) The solution is expected to consist of several steps:
\begin{enumerate}
\item XXX.
\item XXX.
\item XXX.
\

\end{enumerate}
Each of these steps should be clearly explained in the report. 
In all cases, you should explain what you did, why you did it that way, what the results show, what the results mean and put these into context. Evidence and justify your arguments both with relevant citations and with the output of your analysis.

Try to keep the report informative and focussed on the important details and insights -- the report also demonstrates an understanding of what is important. 
There is a maximum page limit of 10 pages, including figures but excluding references, note that this is a limit not a target. All figures must be referenced in the main body of the text, and have captions and must have legible axis labels.

\subsection*{Deliverables}

Hand in via MMS:

\begin{itemize}
\item The bash scripts and the Python code that you write. But do not include any outputted files or linked software.
\item A report in PDF format which contains details of each step of the process, justification for any decisions you take, and an evaluation of the final analysis. This should also contain evidence of functionality (via your results in the report) and any notable figures you have produced with relevant citations throughout.
\end{itemize}

\noindent
Please create a \texttt{.zip} file containing all files and submit this to MMS in the Code3 slot, please also upload your pdf report to the Report3 slot. Please note that MMS is definitive for weighting and deadlines, which, occasionally have to be changed. Ensure you give yourself to upload and download and check that you've uploaded the correct files, do this with enough time to fix if required and re-check before the deadline.


\subsection*{Marking}

This practical will be marked according to the graduate school mark descriptors. All documents relating to the mark descriptors (and their conversion to the 20 point scale) can be found on the graduate school webpages: \url{https://www.st-andrews.ac.uk/graduate-school/students/rules/msc-and-mlitt-documents/}.

For the mark descriptors see: \href{https://www.st-andrews.ac.uk/assets/university/graduate-school/documents/GSIS\%20Mark\%20Descriptors.pdf}{https://www.st-andrews.ac.uk/assets/university/graduate-school/documents/GSIS\%20Mark\%20Descriptors.pdf}. You should also be aware of the following marking guidelines for this practical:

\begin{itemize}

\item To get a mark \textbf{in the 0-3  band} is a submission which shows little evidence of any attempt to complete the work.
\item To get mark \textbf{in the 4-6 band} is a submission which shows little evidence of any acceptable attempt to complete the work, with no substantial relevant material submitted.
\item A \textit{partial working implementation} \textbf{in the 7-10 grade band} is a submission which shows evidence of a reasonable attempt addressing some of the requirements, or is accompanied by a very weak report which does not evidence good understanding.
\item A \textit{basic implementation} \textbf{in the 11-13 band} is a submission which achieves a solution in a straight-forward way and contains some evaluation, but is lacking in quality and detail, or is accompanied by a weaker report which does not evidence good understanding. 
\item An implementation \textbf{in the 14-16 range} should complete all parts of the specification, consist of clean and understandable code, and be accompanied by a good report which clearly describes the process and reasoning behind each step and contains a good discussion of the achieved results and evaluation measures. 
\item To achieve a grade of \textbf{17 and higher}, should have excellent justification and experimentation into the methods used with relevant citations linking with the literature. You must have completed all tasks.
\end{itemize}
Note that the goal is \emph{solid methodology and understanding} rather than 
a collection of extensions -- a good scientific approach and analysis are difficult, whereas running many different algorithms on the same data is easy. Be thorough in your solution and strengthen your basic argument and methodology. 

\noindent
Also note that:

\begin{itemize}
\item We will not focus on software engineering practice and advanced Python 
techniques when marking, but your code should be sensibly organised, commented, and easy to follow. The result (outputs) of your code must be in the pdf report to evidence that your code worked.
\item Overlength penalty: Scheme A, 1 mark for work that is 10\% over-length, then a further 1 mark per additional 10\% over. See \url{https://www.st-andrews.ac.uk/policy/academic-policies-assessment-examination-and-award-coursework-penalties/coursework-penalties.pdf} 
\item
Lateness penalty: 1 mark per 24-hour period, or part thereof. See \url{https://www.st-andrews.ac.uk/policy/academic-policies-assessment-examination-and-award-coursework-penalties/coursework-penalties.pdf} 
\item
Details for good academic practice are outlined on the University webpages here:
\url{https://www.st-andrews.ac.uk/students/rules/academicpractice/}
\item For more details on Graduate School penalties, extension request etc, please refer to \url{https://www.st-andrews.ac.uk/graduate-school/students/rules/}
\item Any use of AI tools, including large language models such as ChatGPT, needs to be acknowledged, referenced, and logged.
If used, text generated by AI should be in quotation marks and referenced as private communication. Code and its comments need to be clearly highlighted and referenced. All
AI interactions used for coding, or for report writing should be annexed to the submission as a searchable text file.


\end{itemize}

\end{document}

